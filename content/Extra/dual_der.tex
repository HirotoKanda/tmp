\newpage
\section{Derivative by Derivative}
\label{sec:dual-der}
\subsection{An Interesting Problem}
\qs{Derivative by derivative}{
  What would the following operator mean?
  \begin{align}
    \odv{}{\odv{}{x}}
  \end{align}
}
Let us define a differential operator $D$:
\dfn{Mysterious Operator}{
  The operator $D$ is defined as the derivative of the derivative with respect to $x$:
  \begin{align}
    D & := \odv{}{x}
  \end{align}
}
Now let us actually apply $D$ to some functions:
\eg{A power function}{
  Let us apply $\odv{}{x}$ to a power function $f(x) = x^n$, $n \in \mathbb{Z}^+$:
  \begin{align}
    D f(x) & = \odv{}{x} x^n = n x^{n-1} = \frac{n}{x} x^n \implies D = \frac{n}{x}
  \end{align}
  Then,
  \begin{align}
    \odv{}{D} f(x) & = \odv{}{D} \pab{\frac{n}{D}}^n = n^n \cdot (-n) \cdot D^{-n-1} = - \pab{\frac{n}{D}}^{n+1} = - x^{n+1}
  \end{align}
}
\eg{Exponential Function}{
  Let us apply $\odv{}{x}$ to the exponential function $f(x) = e^{\alpha x}$:
  \begin{align}
    \odv{f}{x}            & = \odv{}{x} e^{\alpha x} = \alpha f(x) \implies \odv{}{x} = \alpha \\
    \implies \quad D f(x) & = \odv{e^{\alpha x}}{\alpha} = x e^{\alpha x}
  \end{align}
}
An important property of this operator is in the commutator with the $x$ operator:
\dfn{$x$ operator}{
  The $x$ operator is defined as the operator that maps a function $f$ to the function $x \cdot f$:
  \begin{align}
    x: f & \mapsto x \cdot f
  \end{align}
}
\dfn{Commutator}{
  For two operators $A: f \mapsto A f$ and $B: f \mapsto B f$, the commutator is defined as:
  \begin{align}
    \commt{A}{B} & := A B - B A
  \end{align}
}
Then the commutator of the $x$ operator and the $D$ operator is given by:
\begin{align}
  \commt{x}{D} & = x D - D x
\end{align}
applying this to a function $f(x)$, we have:
\begin{align}
  \commt{x}{D} f(x)           & = x D f(x) - D (x f(x))          \\
                              & = x D f(x) - D x f(x) - x D f(x) \\
                              & = - D f(x) = - 1 \cdot f(x)      \\
  \implies \quad \commt{x}{D} & = -1
\end{align}
and notice since both $x$ and $D$ are just linear operators, if we define $x = \odv{}{D}$, then we have:
\begin{align}
  \commt{x}{D} f(D) & = \odv{}{D} D f(D) - D \odv{}{D} f(D) \\
                    & = f(D) + D \odv{f}{D} - D \odv{f}{D}  \\
                    & =  f(D)
\end{align}
so in general,
\begin{align}
  \commt{\omega}{\odv{}{\omega}} & = -1
\end{align}
Note the bi-linearity of the commutator:
\thm{Bi-linearity of Commutator}{
  For $\alpha, \beta \in \cmplx$ and operators $A, B$, the commutator satisfies:
  \begin{align}
    \commt{\alpha A}{\beta B} & = \alpha \beta \commt{A}{B}
  \end{align}
}
This immidiately leads to:
\prcp{Canonical Commutation Relation}{
  For the $x$ operator and the $D$ operator, the commutation relation is given by:
  \begin{align}
    \commt{x}{-i \hbar D} & = i \hbar = \commt{x}{-i\hbar \pdv{}{x}} = \commt{i \hbar \pdv{}{p}}{p}
  \end{align}
  \label{prcp:canonical-commutation-relation}
}


Let us remark on other properties:
\thm{Commutator of General Function}{
  For a general function $f(x)$, the commutator with the $D$ operator is given by:
  \begin{align}
    \commt{D}{f(x)} & = \odv{f(x)}{x}
  \end{align}
}
\pf{Proof}{
  Consider functions $f(x)$ and $g(x)$, then:
  \begin{align}
    \commt{D}{f(x)} g(x)     & = D f(x) g(x) - f(x) D g(x)                                    \\
                             & = \odv{f(x)}{x} g(x) + f(x) \odv{g(x)}{x} - f(x) \odv{g(x)}{x} \\
                             & = \odv{f(x)}{x} g(x)                                           \\
    \implies \commt{D}{f(x)} & = \odv{f(x)}{x}
  \end{align}
}
\thm{Commutator of General Operator}{
  For a general operator $F$ that is a function of $D$, the commutator with the $x$ operator is given by:
  \begin{align}
    \commt{F(D)}{x} & = \odv{F(D)}{D}
  \end{align}
}
\pf{Proof}{
  Assume that $F$ is a function of $D$, and we can write it as a Taylor series:
  \begin{align}
    F(D) & = \sum_{n=0}^{\infty} a_n D^n
  \end{align}
  Then,
  \begin{align}
    \commt{F}{x} & = F x  - x F                                                    \\
                 & = \sum_{n=0}^{\infty} a_n D^n x - x \sum_{n=0}^{\infty} a_n D^n \\
                 & = \sum_{n=0}^{\infty} a_n (D^n x - x D^n)                       \\
                 & = \sum_{n=0}^{\infty} a_n n D^{n-1} = \odv{F}{D}
  \end{align}
}

Finally, let us comment on the relationship with the \emph{shift operator}:
\dfn{Shift Operator}{
  The \emph{shift operator} $S_a$ is defined as the operator that shifts a function by a constant $a$:
  \begin{align}
    S_a [f](x) & := f(x + a)
  \end{align}
}
The commutator of the shift operator and the $x$ operator is given by:
\begin{align}
  \commt{S_a}{x}f(x) & = S_a [x f(x)] - x S_a [f(x)]   \\
                     & = (x + a) f(x + a) - x f(x + a) \\
                     & = a f(x + a) = a S_a [f(x)]
\end{align}
Thus,
\thm{Commutator of Shift Operator and $x$}{
  The commutator of the shift operator $S_a$ and the $x$ operator is given by:
  \begin{align}
    \commt{S_a}{x} & = a S_a \implies S_a = e^{a D}
  \end{align}
}
