\section{Lagrange Formalism}
\subsection{Quick Recap: Newtonian Mechanics}
In Newtonian mechanics, the motion of a particle is described through a few important quantities:
for a particle of (inertial) mass $m$, position $\vec{r}$, we have
\begin{align}
  \text{velocity}: \vec{v}     & = \odv{\vec{r}}{t}                       \\
  \text{acceleration}: \vec{a} & = \odv{\vec{v}}{t} = \odv[2]{\vec{r}}{t} \\
  \text{momentum}: \vec{p}     & = m\vec{v} = m\odv{\vec{r}}{t}
\end{align}
and the relations between these quantities, in the presence of external forces $\vec{F}_{\text{ext}}^{(i)}$ acting on the particle, are given by \textit{Newton's second law}:
\begin{align}
  \odv{\vec{p}}{t} = m \odv[2]{\vec{r}}{t} = \sum_i \vec{F}_{\text{ext}}^{(i)}
\end{align}
The work done by such forces is given by
\begin{align}
  W_{\text{total}} & = \sum_i \int_l \odif{\vec{x}} \cdot \vec{F}_{\text{ext}}^{(i)}, \quad \text{where } l \text{ is the path of the particle.}
\end{align}
This is the energy change of the particle through the motion:
\begin{align}
  W_{\text{total}}
   & = \sum_i \int_l \odif{\vec{x}} \cdot \vec{F}_{\text{ext}}^{(i)} \\
   & = \int_{t_i}^{t_f} \odif{t} \, \vec{v} \cdot m \odv{\vec{v}}{t} \\
   & = \int_{t_i}^{t_f} \odif{t} \, \frac{m}{2} \odv{}{t} \vec{v}^2  \\
   & = \frac{m}{2} \vec{v}_f^2 - \frac{m}{2} \vec{v}_i^2
\end{align}
meaning that $m \vec{v}^2 / 2$ is the energy due to the motion of the particle: the \textbf{kinetic energy} $T$:
\begin{align}
  T & = \frac{m}{2} \vec{v}^2
\end{align}
Now, often, the external force acting on the particle is due to a potential $V$:
\begin{align}
  \vec{F}_{\text{ext}} & = -\nabla V
\end{align}
For example, for a 1D spring, the potential is given by
\eg{1D spring/ Harmonic potential}{
  \vspace*{-2em}
  \begin{align}
    V & = \frac{1}{2} k x^2 \implies F_{\text{ext}} = -k x
  \end{align}
}
or the electrostatic potential:
\eg{Electrostatic potential}{
  \vspace*{-2em}
  \begin{align}
    V & = \frac{1}{4 \pi \varepsilon_0} \frac{q_1 q_2}{r} \implies F_{\text{ext}} = -\nabla V = -\frac{q_1 q_2}{4 \pi \varepsilon_0 r^2} \hat{r}
  \end{align}
}

\subsection{Lagrangian and Euler-Lagrange Equation}
Let us define quantities called the \textbf{Lagrangian} $L$ and \textbf{action} $S$:
\begin{align}
  L & := T - V, \qquad  S := \int_{t_i}^{t_f} \odif{t} \, L
\end{align}
For one particle, the Lagrangian in general contains the position $q_i(t)$ and velocity $\dot{q_i}(t)$ and time $t$:
\begin{align}
  L & = L(q_i, \dot{q_i}, t)
\end{align}
The action is then given by
\begin{align}
  S & = \int_{t_i}^{t_f} \odif{t} \, L(q_i(t), \dot{q_i}(t), t)
\end{align}

