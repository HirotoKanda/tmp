\section{Historical Background}
\subsection{de Broglie Hypothesis}
In 1905, Albert Einstein found the photoelectric effect, suggesting that light has a particle like property, such as discrete energy
\cite{1905-photoelectric}.
Following this work, Arthur Compton in 1923 discovered the Compton effect, which is the scattering of X-rays by electrons, further supporting the particle nature of light and confirming the discrete momentum of photons
\cite{1923-compton}.
Specifically, energy and momentum are each related to the frequency and wavelength of light, respectively, as follows:
\mprop{Planck-Einstein relation}{
  For a photon with frequency $\nu$ and wave length $\lambda$, the energy $E$ and momentum $p$ are given by
  \begin{align}
    E = h \nu, \quad p = \frac{h}{\lambda} = \frac{h \nu}{c} = \frac{E}{c}, \label{eq:photon-energy-momentum}
  \end{align}
}
In 1925, Louis de Brogile posulated that other particles (or any matter) also have a wave-like property, and the energy-frequency/ momentum-wavelength relation is given by the Planck-Einstein relation Eq. \eqref{eq:photon-energy-momentum} \cite{1925-deBroglie}.

\subsection{Bohr Model}
In 1913, Niels Bohr proposed a model of the hydrogen atom, which describes the electron as a particle orbiting the nucleus in discrete energy levels \cite{1913-bohr}.
The implication of this model is that electrons have a fixed, discrete(quantized) angular momentum $\vec{L} := \vec{r} \times \vec{p}$.
For a particle orbiting in a circular orbit, the angular momentum is given by
\begin{align}
  \vec{L} & = r p = \frac{h r}{\lambda}
\end{align}
For the electron wave to be continous around the orbit of radius $r$, the wavelength must be an integer multiple of the circumference of the orbit:
\begin{align}
  \lambda & = \frac{2 \pi r}{n}, \quad n = 1, 2, 3, \ldots \implies \vec{L} = n \frac{h}{2 \pi} := n \hbar
\end{align}


\section{Canonical Quantization}


\section{Bra-ket Notation and Operator Formalism}


\section{Schrödinger Equation}

\section{Harmonic Oscillator}