\section{Hilbert Space in Quantum Mechanics}
\subsection{Hilbert Space}
\dfn{Hilbert Space}{
  A \emph{Hilbert space} is a complete inner product space, which is a vector space with an inner product that is complete with respect to the norm induced by the inner product.
  It is denoted as $\hilbert$.
  We denote the basis vectors of the Hilbert space as $\ket{i}$, where $i$ is an index.
}
\dfn{(Hermitian) Inner Product and Norm}{
  \emph{Inner product} of a Hilbert space $\hilbert$ is a function $\hilbert \times \hilbert \to \mathbb{C}$,
  that satisfies the following properties:
  \begin{enumerate}
    \item Conjugate symmetry
          \begin{align}
            \braket{\psi}{\phi} = \braket{\phi}{\psi}^*
          \end{align}
    \item Linearity
          \begin{align}
            \bra{\psi}(a \ket{\varphi} + b \ket{\phi}) = a \braket{\psi}{\varphi} + b \braket{\psi}{\phi}
          \end{align}
    \item Positivity
          \begin{align}
            \braket{\psi}{\psi} \geq 0, \braket{\psi}{\psi} = 0 \iff \ket{\psi} = \zeroket
          \end{align}
  \end{enumerate}
  where $\ket{\psi}, \ket{\phi}, \ket{\varphi} \in \hilbert$ and $a, b \in \mathbb{C}$.

  The \emph{norm} of a vector $\ket{\psi} \in \hilbert$ is defined as
  \begin{align}
    \norm{\ket{\psi}} & := \sqrt{\braket{\psi}{\psi}} \geq 0
  \end{align}
}
\dfn{Completeness}{
  A vector space is said to be \emph{complete} if every Cauchy sequence in the space converges to a limit in the space:
  \begin{align}
    \forall \epsilon > 0, \exists N \in \mathbb{N} : \forall m, n > N, \norm{\ket{v_m} - \ket{v_n}} < \epsilon
  \end{align}
  where $\ket{v_m}, \ket{v_n} \in \hilbert$ are vectors in the space.
  Colloquially, this property ensures that any sum(even uncountably infinite) of vectors in $\hilbert$ will still be in $\hilbert$.
}

\subsection{Basis Vectors and Completeness Relation}
\dfn{Basis Vectors}{
  A set of vectors $\{\ket{i}\}$ in a Hilbert space $\hilbert$ is said to be a \emph{basis} if it is a set of vectors that spans the space, satisfying the following conditions:
  \begin{enumerate}
    \item Linear Independence
          \begin{align}
            \sum_i c_i \ket{i} & = \zeroket \iff c_i = 0 \quad \forall i
          \end{align}
    \item Completeness
          \begin{align}
            ^{\forall} \ket{\psi} \in \hilbert, ^{\exists} \{c_i\} \in \mathbb{C} \text{ s.t. } \ket{\psi} = \sum_i c_i \ket{i}
          \end{align}
    \item Orthogonality
          \begin{align}
            \braket{i}{j} & = 0 \quad \forall i \neq j
          \end{align}
  \end{enumerate}
  Especially, if the basis vectors satisfy:
  \begin{align}
    \braket{i}{j} & = \delta_{ij} \quad \forall i, j
  \end{align}
  then the basis is said to be \emph{orthonormal}.
}
\thm{Completeness Relation}{
  The following relation holds for a complete set of orthonormal basis vectors $\{\ket{i}\}$ in a Hilbert space $\hilbert$:
  \begin{align}
    \ket{\psi} = \sum_i c_i \ket{i} \iff \sum_i \ketbra{i}{i} & = \idty
  \end{align}
}
\pf{Proof}{
  \begin{itemize}
    \item $(\implies)$
          \begin{align}
            \sum_i \ket{i} \braket{i}{\psi} & = \sum_i c_i \ket{i} = \ket{\psi}, \quad c_i := \braket{i}{\psi}
          \end{align}
    \item $(\impliedby)$
          \begin{align}
            \ket{\psi} & = \idty \ket{\psi} = \sum_i \ket{i} \braket{i}{\psi} = \sum_i c_i \ket{i}
          \end{align}
  \end{itemize}
}